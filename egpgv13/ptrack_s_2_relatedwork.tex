\section{Related Work}

\subsection{Feature Extraction and Tracking}

Feature extraction and tracking are two closely related problems in feature-based visualization. Conventional approaches extract features from individual time steps and then associate them between consecutive time steps. Silver and Wang \cite{Silver1997} defined threshold connected components as their features, and tracked overlapped features by calculating their differences. Reinders~\cite{Reinders2001} introduced a prediction verification tracking technique that calculates a prediction by linear extrapolation based on the previous feature path, and a candidate will be added to the path if it corresponds to that prediction. Theisel and Seidel \cite{Theisel2003} represented dynamic behavior of features as steam lines of critical points in a higher dimensional vector field, such that no correspondence analysis of features in consecutive time step is required. Ji and Shen \cite{Ji2003} introduced a method to track local features from time-varying data using higher-dimensional iso-surfacing. They also used a global optimization correspondence algorithm to improve the robustness of feature tracking \cite{Ji2006}. Caban et al. \cite{Caban2007} estimated a tracking window and compared feature distance of textural properties to find the best match within the window. Bremer et al. \cite{Bremer2007} described two topological feature tracking methods where one employs Jacobi sets to track critical points and the other uses distance measures on graphs to track channel structures. Muelder and Ma \cite{Muelder2009} introduced a prediction-correction approach that first predicts a feature region based on the centroid location of that in the previous time steps, and then corrects the predicted region by adjusting the surface boundaries via region growing and shrinking. This approach is appealing for its computing efficiency and the reliability in an interactive system. Ozer and Wei presented a group feature tracking framework\cite{Ozer2012} that clusters features based on similarity measures and tracks features of similar behavior in groups. However, it is difficult to obtain the global feature descriptions if a single processing node cannot hold the whole volume, unless the descriptions can be shared and merged in an efficient way.

\subsection{Parallel Feature Extraction and Tracking}

To boost the speed for feature tracking in data-distributed applications, Chen et al.~\cite{Chen03realtime} developed a two-stage partial-merge strategy using the master-slave paradigm. The slaves first exchange local connectivity information using Binary-tree merge, and then a visualization host collects and correlates the local information to generate the global connectivity. This approach is not scalable since half of the processors will become idle after each merge. It is also unclear how the host can efficiently collect local connectivity information from the slaves, since gathering operation can be expensive given a large number of processors.

\subsection{Parallel Graph Algorithm and Applications}

Graph-based algorithms have long been studied and used for a wide range of applications, typically along the line of divide-and-conquer approaches. Grundmann et al.~\cite{36247} introduced a hierarchical graph-based approach for video segmentation, a closely related research topic to 3D flow feature extraction as video can be treated as a space-time volume of image data~\cite{Klein2002}. In their work, a connected sequence of time-axis-aligned subsets of cubic image volumes are assigned to a set of corresponding processors, and the incident regions are merged if they are inside the volumes window. The incident regions on window boundary are first marked as ambiguous and later connected by merging the neighboring windows to a larger window, which consists of the unresolved regions form both windows on their common boundary. This approach is not applicable for a memory intensive situation since the allocated volume size before merging might already reache the memory capacity. Liu and Sun \cite{Liu2010} made a parallelization of the graph-cuts optimization algorithm~\cite{Boykov2004}, in which data are uniformly partitioned and then are adaptively merged to achieve fast graph-cuts. These approaches are suitable for shared-memory but not message-passing parallelization due to their frequent shifting on data ranges.