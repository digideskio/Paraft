
%% bare_conf.tex
%% V1.3
%% 2007/01/11
%% by Michael Shell
%% See:
%% http://www.michaelshell.org/
%% for current contact information.
%%
%% This is a skeleton file demonstrating the use of IEEEtran.cls
%% (requires IEEEtran.cls version 1.7 or later) with an IEEE conference paper.
%%
%% Support sites:
%% http://www.michaelshell.org/tex/ieeetran/
%% http://www.ctan.org/tex-archive/macros/latex/contrib/IEEEtran/
%% and
%% http://www.ieee.org/

%%*************************************************************************
%% Legal Notice:
%% This code is offered as-is without any warranty either expressed or
%% implied; without even the implied warranty of MERCHANTABILITY or
%% FITNESS FOR A PARTICULAR PURPOSE! 
%% User assumes all risk.
%% In no event shall IEEE or any contributor to this code be liable for
%% any damages or losses, including, but not limited to, incidental,
%% consequential, or any other damages, resulting from the use or misuse
%% of any information contained here.
%%
%% All comments are the opinions of their respective authors and are not
%% necessarily endorsed by the IEEE.
%%
%% This work is distributed under the LaTeX Project Public License (LPPL)
%% ( http://www.latex-project.org/ ) version 1.3, and may be freely used,
%% distributed and modified. A copy of the LPPL, version 1.3, is included
%% in the base LaTeX documentation of all distributions of LaTeX released
%% 2003/12/01 or later.
%% Retain all contribution notices and credits.
%% ** Modified files should be clearly indicated as such, including  **
%% ** renaming them and changing author support contact information. **
%%
%% File list of work: IEEEtran.cls, IEEEtran_HOWTO.pdf, bare_adv.tex,
%%                    bare_conf.tex, bare_jrnl.tex, bare_jrnl_compsoc.tex
%%*************************************************************************

% *** Authors should verify (and, if needed, correct) their LaTeX system  ***
% *** with the testflow diagnostic prior to trusting their LaTeX platform ***
% *** with production work. IEEE's font choices can trigger bugs that do  ***
% *** not appear when using other class files.                            ***
% The testflow support page is at:
% http://www.michaelshell.org/tex/testflow/



% Note that the a4paper option is mainly intended so that authors in
% countries using A4 can easily print to A4 and see how their papers will
% look in print - the typesetting of the document will not typically be
% affected with changes in paper size (but the bottom and side margins will).
% Use the testflow package mentioned above to verify correct handling of
% both paper sizes by the user's LaTeX system.
%
% Also note that the "draftcls" or "draftclsnofoot", not "draft", option
% should be used if it is desired that the figures are to be displayed in
% draft mode.
%
\documentclass[10pt, conference, compsocconf]{IEEEtran}
% Add the compsocconf option for Computer Society conferences.
%
% If IEEEtran.cls has not been installed into the LaTeX system files,
% manually specify the path to it like:
% \documentclass[conference]{../sty/IEEEtran}





% Some very useful LaTeX packages include:
% (uncomment the ones you want to load)


% *** MISC UTILITY PACKAGES ***
%
%\usepackage{ifpdf}
% Heiko Oberdiek's ifpdf.sty is very useful if you need conditional
% compilation based on whether the output is pdf or dvi.
% usage:
% \ifpdf
%   % pdf code
% \else
%   % dvi code
% \fi
% The latest version of ifpdf.sty can be obtained from:
% http://www.ctan.org/tex-archive/macros/latex/contrib/oberdiek/
% Also, note that IEEEtran.cls V1.7 and later provides a builtin
% \ifCLASSINFOpdf conditional that works the same way.
% When switching from latex to pdflatex and vice-versa, the compiler may
% have to be run twice to clear warning/error messages.






% *** CITATION PACKAGES ***
%
%\usepackage{cite}
% cite.sty was written by Donald Arseneau
% V1.6 and later of IEEEtran pre-defines the format of the cite.sty package
% \cite{} output to follow that of IEEE. Loading the cite package will
% result in citation numbers being automatically sorted and properly
% "compressed/ranged". e.g., [1], [9], [2], [7], [5], [6] without using
% cite.sty will become [1], [2], [5]--[7], [9] using cite.sty. cite.sty's
% \cite will automatically add leading space, if needed. Use cite.sty's
% noadjust option (cite.sty V3.8 and later) if you want to turn this off.
% cite.sty is already installed on most LaTeX systems. Be sure and use
% version 4.0 (2003-05-27) and later if using hyperref.sty. cite.sty does
% not currently provide for hyperlinked citations.
% The latest version can be obtained at:
% http://www.ctan.org/tex-archive/macros/latex/contrib/cite/
% The documentation is contained in the cite.sty file itself.






% *** GRAPHICS RELATED PACKAGES ***
%
\ifCLASSINFOpdf
  % \usepackage[pdftex]{graphicx}
  % declare the path(s) where your graphic files are
  % \graphicspath{{../pdf/}{../jpeg/}}
  % and their extensions so you won't have to specify these with
  % every instance of \includegraphics
  % \DeclareGraphicsExtensions{.pdf,.jpeg,.png}
\else
  % or other class option (dvipsone, dvipdf, if not using dvips). graphicx
  % will default to the driver specified in the system graphics.cfg if no
  % driver is specified.
  % \usepackage[dvips]{graphicx}
  % declare the path(s) where your graphic files are
  % \graphicspath{{../eps/}}
  % and their extensions so you won't have to specify these with
  % every instance of \includegraphics
  % \DeclareGraphicsExtensions{.eps}
\fi
% graphicx was written by David Carlisle and Sebastian Rahtz. It is
% required if you want graphics, photos, etc. graphicx.sty is already
% installed on most LaTeX systems. The latest version and documentation can
% be obtained at: 
% http://www.ctan.org/tex-archive/macros/latex/required/graphics/
% Another good source of documentation is "Using Imported Graphics in
% LaTeX2e" by Keith Reckdahl which can be found as epslatex.ps or
% epslatex.pdf at: http://www.ctan.org/tex-archive/info/
%
% latex, and pdflatex in dvi mode, support graphics in encapsulated
% postscript (.eps) format. pdflatex in pdf mode supports graphics
% in .pdf, .jpeg, .png and .mps (metapost) formats. Users should ensure
% that all non-photo figures use a vector format (.eps, .pdf, .mps) and
% not a bitmapped formats (.jpeg, .png). IEEE frowns on bitmapped formats
% which can result in "jaggedy"/blurry rendering of lines and letters as
% well as large increases in file sizes.
%
% You can find documentation about the pdfTeX application at:
% http://www.tug.org/applications/pdftex





% *** MATH PACKAGES ***
%
%\usepackage[cmex10]{amsmath}
% A popular package from the American Mathematical Society that provides
% many useful and powerful commands for dealing with mathematics. If using
% it, be sure to load this package with the cmex10 option to ensure that
% only type 1 fonts will utilized at all point sizes. Without this option,
% it is possible that some math symbols, particularly those within
% footnotes, will be rendered in bitmap form which will result in a
% document that can not be IEEE Xplore compliant!
%
% Also, note that the amsmath package sets \interdisplaylinepenalty to 10000
% thus preventing page breaks from occurring within multiline equations. Use:
%\interdisplaylinepenalty=2500
% after loading amsmath to restore such page breaks as IEEEtran.cls normally
% does. amsmath.sty is already installed on most LaTeX systems. The latest
% version and documentation can be obtained at:
% http://www.ctan.org/tex-archive/macros/latex/required/amslatex/math/





% *** SPECIALIZED LIST PACKAGES ***
%
%\usepackage{algorithmic}
% algorithmic.sty was written by Peter Williams and Rogerio Brito.
% This package provides an algorithmic environment fo describing algorithms.
% You can use the algorithmic environment in-text or within a figure
% environment to provide for a floating algorithm. Do NOT use the algorithm
% floating environment provided by algorithm.sty (by the same authors) or
% algorithm2e.sty (by Christophe Fiorio) as IEEE does not use dedicated
% algorithm float types and packages that provide these will not provide
% correct IEEE style captions. The latest version and documentation of
% algorithmic.sty can be obtained at:
% http://www.ctan.org/tex-archive/macros/latex/contrib/algorithms/
% There is also a support site at:
% http://algorithms.berlios.de/index.html
% Also of interest may be the (relatively newer and more customizable)
% algorithmicx.sty package by Szasz Janos:
% http://www.ctan.org/tex-archive/macros/latex/contrib/algorithmicx/




% *** ALIGNMENT PACKAGES ***
%
%\usepackage{array}
% Frank Mittelbach's and David Carlisle's array.sty patches and improves
% the standard LaTeX2e array and tabular environments to provide better
% appearance and additional user controls. As the default LaTeX2e table
% generation code is lacking to the point of almost being broken with
% respect to the quality of the end results, all users are strongly
% advised to use an enhanced (at the very least that provided by array.sty)
% set of table tools. array.sty is already installed on most systems. The
% latest version and documentation can be obtained at:
% http://www.ctan.org/tex-archive/macros/latex/required/tools/


%\usepackage{mdwmath}
%\usepackage{mdwtab}
% Also highly recommended is Mark Wooding's extremely powerful MDW tools,
% especially mdwmath.sty and mdwtab.sty which are used to format equations
% and tables, respectively. The MDWtools set is already installed on most
% LaTeX systems. The lastest version and documentation is available at:
% http://www.ctan.org/tex-archive/macros/latex/contrib/mdwtools/


% IEEEtran contains the IEEEeqnarray family of commands that can be used to
% generate multiline equations as well as matrices, tables, etc., of high
% quality.


%\usepackage{eqparbox}
% Also of notable interest is Scott Pakin's eqparbox package for creating
% (automatically sized) equal width boxes - aka "natural width parboxes".
% Available at:
% http://www.ctan.org/tex-archive/macros/latex/contrib/eqparbox/





% *** SUBFIGURE PACKAGES ***
%\usepackage[tight,footnotesize]{subfigure}
% subfigure.sty was written by Steven Douglas Cochran. This package makes it
% easy to put subfigures in your figures. e.g., "Figure 1a and 1b". For IEEE
% work, it is a good idea to load it with the tight package option to reduce
% the amount of white space around the subfigures. subfigure.sty is already
% installed on most LaTeX systems. The latest version and documentation can
% be obtained at:
% http://www.ctan.org/tex-archive/obsolete/macros/latex/contrib/subfigure/
% subfigure.sty has been superceeded by subfig.sty.



%\usepackage[caption=false]{caption}
%\usepackage[font=footnotesize]{subfig}
% subfig.sty, also written by Steven Douglas Cochran, is the modern
% replacement for subfigure.sty. However, subfig.sty requires and
% automatically loads Axel Sommerfeldt's caption.sty which will override
% IEEEtran.cls handling of captions and this will result in nonIEEE style
% figure/table captions. To prevent this problem, be sure and preload
% caption.sty with its "caption=false" package option. This is will preserve
% IEEEtran.cls handing of captions. Version 1.3 (2005/06/28) and later 
% (recommended due to many improvements over 1.2) of subfig.sty supports
% the caption=false option directly:
%\usepackage[caption=false,font=footnotesize]{subfig}
%
% The latest version and documentation can be obtained at:
% http://www.ctan.org/tex-archive/macros/latex/contrib/subfig/
% The latest version and documentation of caption.sty can be obtained at:
% http://www.ctan.org/tex-archive/macros/latex/contrib/caption/




% *** FLOAT PACKAGES ***
%
%\usepackage{fixltx2e}
% fixltx2e, the successor to the earlier fix2col.sty, was written by
% Frank Mittelbach and David Carlisle. This package corrects a few problems
% in the LaTeX2e kernel, the most notable of which is that in current
% LaTeX2e releases, the ordering of single and double column floats is not
% guaranteed to be preserved. Thus, an unpatched LaTeX2e can allow a
% single column figure to be placed prior to an earlier double column
% figure. The latest version and documentation can be found at:
% http://www.ctan.org/tex-archive/macros/latex/base/



%\usepackage{stfloats}
% stfloats.sty was written by Sigitas Tolusis. This package gives LaTeX2e
% the ability to do double column floats at the bottom of the page as well
% as the top. (e.g., "\begin{figure*}[!b]" is not normally possible in
% LaTeX2e). It also provides a command:
%\fnbelowfloat
% to enable the placement of footnotes below bottom floats (the standard
% LaTeX2e kernel puts them above bottom floats). This is an invasive package
% which rewrites many portions of the LaTeX2e float routines. It may not work
% with other packages that modify the LaTeX2e float routines. The latest
% version and documentation can be obtained at:
% http://www.ctan.org/tex-archive/macros/latex/contrib/sttools/
% Documentation is contained in the stfloats.sty comments as well as in the
% presfull.pdf file. Do not use the stfloats baselinefloat ability as IEEE
% does not allow \baselineskip to stretch. Authors submitting work to the
% IEEE should note that IEEE rarely uses double column equations and
% that authors should try to avoid such use. Do not be tempted to use the
% cuted.sty or midfloat.sty packages (also by Sigitas Tolusis) as IEEE does
% not format its papers in such ways.





% *** PDF, URL AND HYPERLINK PACKAGES ***
%
%\usepackage{url}
% url.sty was written by Donald Arseneau. It provides better support for
% handling and breaking URLs. url.sty is already installed on most LaTeX
% systems. The latest version can be obtained at:
% http://www.ctan.org/tex-archive/macros/latex/contrib/misc/
% Read the url.sty source comments for usage information. Basically,
% \url{my_url_here}.





% *** Do not adjust lengths that control margins, column widths, etc. ***
% *** Do not use packages that alter fonts (such as pslatex).         ***
% There should be no need to do such things with IEEEtran.cls V1.6 and later.
% (Unless specifically asked to do so by the journal or conference you plan
% to submit to, of course. )


% correct bad hyphenation here
\hyphenation{op-tical net-works semi-conduc-tor}


\begin{document}
%
% paper title
% can use linebreaks \\ within to get better formatting as desired
\title{Parallel Feature Extraction and Tracking for 3D Vortical Data (TBD)}


% author names and affiliations
% use a multiple column layout for up to two different
% affiliations

\author{\IEEEauthorblockN{Authors Name/s per 1st Affiliation (Author)}
\IEEEauthorblockA{line 1 (of Affiliation): dept. name of organization\\
line 2: name of organization, acronyms acceptable\\
line 3: City, Country\\
line 4: Email: name@xyz.com}
\and
\IEEEauthorblockN{Authors Name/s per 2nd Affiliation (Author)}
\IEEEauthorblockA{line 1 (of Affiliation): dept. name of organization\\
line 2: name of organization, acronyms acceptable\\
line 3: City, Country\\
line 4: Email: name@xyz.com}
}

% conference papers do not typically use \thanks and this command
% is locked out in conference mode. If really needed, such as for
% the acknowledgment of grants, issue a \IEEEoverridecommandlockouts
% after \documentclass

% for over three affiliations, or if they all won't fit within the width
% of the page, use this alternative format:
% 
%\author{\IEEEauthorblockN{Michael Shell\IEEEauthorrefmark{1},
%Homer Simpson\IEEEauthorrefmark{2},
%James Kirk\IEEEauthorrefmark{3}, 
%Montgomery Scott\IEEEauthorrefmark{3} and
%Eldon Tyrell\IEEEauthorrefmark{4}}
%\IEEEauthorblockA{\IEEEauthorrefmark{1}School of Electrical and Computer Engineering\\
%Georgia Institute of Technology,
%Atlanta, Georgia 30332--0250\\ Email: see http://www.michaelshell.org/contact.html}
%\IEEEauthorblockA{\IEEEauthorrefmark{2}Twentieth Century Fox, Springfield, USA\\
%Email: homer@thesimpsons.com}
%\IEEEauthorblockA{\IEEEauthorrefmark{3}Starfleet Academy, San Francisco, California 96678-2391\\
%Telephone: (800) 555--1212, Fax: (888) 555--1212}
%\IEEEauthorblockA{\IEEEauthorrefmark{4}Tyrell Inc., 123 Replicant Street, Los Angeles, California 90210--4321}}




% use for special paper notices
%\IEEEspecialpapernotice{(Invited Paper)}




% make the title area
\maketitle


\begin{abstract}
A large-scale time-varying flow data set can take terabytes to petabytes of storage space. One promising approach is to extract features of interest and store only those features. This requires storage space that is several orders of magnitude smaller than the raw data would take. However, extracting and tracking 3D features is a non-trivial task since features from the global perspective are separated within different processing nodes. In this paper, we present a approach to contract the connectivity information of features reside in separated nodes, efficiently. Utilizing the connectivity information, feature extraction and tracking can be done in parallel as they were done in a single node. We demonstrate the effectiveness of this method with three vortical flow datasets and the scalability of our system on parallel and distributed systems.
\end{abstract}

\begin{IEEEkeywords}
component; formatting; style; styling;
\end{IEEEkeywords}


% For peer review papers, you can put extra information on the cover
% page as needed:
% \ifCLASSOPTIONpeerreview
% \begin{center} \bfseries EDICS Category: 3-BBND \end{center}
% \fi
%
% For peerreview papers, this IEEEtran command inserts a page break and
% creates the second title. It will be ignored for other modes.
\IEEEpeerreviewmaketitle



\section{Introduction}
% no \IEEEPARstart
The increasing computational power and accessibility to supercomputers have brought scientists the capability to simulate physical phenomena of unprecedented complexity at high spatial and temporal resolution. However, these large-scale time-varying data sets can take gigabyte or even terabytes of space to preserve. One promising solution to the problem is to introduce feature extraction and tracking techniques. Instead of duplicating raw data along the exploring process, dealing with features of interest requires memory space that is several orders of magnitude smaller than the raw data would take.
However, extracting and tracking features in distributed vortical datasets is a non-trivial task. Existing researches on feature-based data visualization have been done mostly focusing on decomposing features using quantitative measures such as size, location, shape or topology information, etc. These measures cannot be applied to distribute volume data directly since vortical features are consist of certain amount of voxels and are very likely to span over blocks as they evolve over time. Therefore existing quantitative measures of partial data scattered in different data blocks cannot be used to describe an integrated feature, unless the distribution of partial features can be obtained beforehand. 

To obtain the feature distribution, a connectivity map of each feature should be generated and maintained. In this paper, we proposed an approach for creating and maintaining feature residual, as well as connectivity information utilizing parallel graphs. Comparing to previous approaches, which generates and maintains the global features information in a single host node, our approach can be done locally that only involves residual data blocks of target features. This requires least communication overhead and avoids the potential bottleneck on the host node. We demonstrate the effectiveness of this method with three vortical flow datasets and the scalability of our system in a distributed environment.

% You must have at least 2 lines in the paragraph with the drop letter
% (should never be an issue)

\section{Related Work}
Extraction and tracking are two closely related problems in feature-based visualization. Although many feature tracking algorithms have been introduced, most of them extract features from single time steps and then try to associate them between consecutive time steps. Silver and Wang [26] considered threshold connected components as their features, and tracked overlapped features between successive time steps by calculating their differences. Octree was employed in their method to speed up the performance and the criteria they used were domain dependent. Reinders et al. [24] introduced a prediction verification tracking technique that calculates a prediction by linear extrapolation based on the previous feature path, an candidate will be added to the path if it corresponds to that prediction. Ji and Shen [17] introduced a method to track local features from time-varying data by using higher-dimensional isosurfacing. They also used a global optimization correspondence algorithm to improve the robustness of feature tracking [16]. Caban et al. [3] estimated  a tracking window and then tried to find the best match between that window and different sub-volumes of subsequence frames by comparing their distance of textural properties.  Also, Bremer et al. [1] described two topological feature tracking methods, one employs Jacobi sets to track critical points while another uses distance measures on graphs to track channel structures. 

Most of the above methods extract features from each time step independently and then apply the correspondence calculation. This could be very slow especially when the size of dataset becomes large. Muelder and Ma [19] introduced a prediction-correction approach that first predicts feature region based on the centroid location of previous time steps, then correcting the predicted region by adjusting the surface boundaries via region growing and shrinking. This approach is appealing because of its computing efficiency and the reliability in an interactive system.

From the parallel processing side of perspective, most of the existing feature extraction and tracking approaches rely on some form of global feature information, while in distributed environment must be communicated among processors to obtain. Chen and Silver [] introduced a two stage partial-merge strategy, exchanging local connectivity information using Binary-tree merge, then a visualization server correlate local data to create a global data. This approach is not scalable since half of the processors will become idle after each merge. It is also unclear that how the visualization server can efficiently collect local connectivity information from non-server processor, since gathering operation is typically very expensive given a large number of processors. 

The approach proposed in this paper follows a different paradigm. Instead of sending local connective information back to the host, the local connectivity information are computed and preserved only in nodes where correspondent feature resides in. There is no global connectivity information preserved in the host node and it only acts as the interface from where the criterion of feature of interest is broadcasted. By doing this, the computation of merging local connectivity information is distributed to those non-host nodes and hence effectively reduces the potential communication bottleneck on the host node. What's more, there's no need to set a barrier and wait before all connectivity information was send back to the host and thus if there exists features that span over a large number of nodes but was not selected by the user, the potentially long computation time for these features will not block the whole process. This makes it ideal for an interactive system, where users can select the feature of interest and instantly receive the visual feedback as the feature evolves.

\section{Algorithm Design}
Though features can be extracted using the afore-mentioned methods within individual processors, some of them may span over multiple data blocks, which is unavoidable as the number of processors increases. 

To perform feature extraction and tracking over a distributed volume dataset, connectivity graphs from either global or local perspective should be constructed. Since meta-info such as feature size, curvature or velocity of partial features are not shared among processors, the connectivity information makes it possible to gather from each part of the feature that resides in different nodes these meta-info so that feature tracking and similarity test could be achieved. 

However, constructing connectivity graphs requires exchanging of local feature information and hence will introduce extra communication cost. To design a proper communication schema for better performance and scalability, the following three factors should be carefully considered:
1.	How many communications are required to complete the connectivity graph;
2.	How many processors will be involved in one communication;
3.	The amount of data being sent and received in one communication.
In the following sections, we will give a detailed explanation on how to create and maintain the connectivity graph with minimum cost over the above three factors.

\subsection{Feature Extraction}
Feature extraction is a process that first detects features, then calculates quantitative attributes describing its characteristics. In general, features can be any interesting pattern, structure, or object that are considered relevant for the investigation of an application and they can be extracted by common technique such as region growing, geometry or topology based clustering method, or other domain specific algorithms [TODO]. 

In our system, a standard region-growing algorithm [14] is used to partition the original volume data into an initial set of features. This can be done by first spreading a set of seeding points inside the volume in a breath-first fashion, and then categorizing voxels into separate regions, each regarded as a single feature. Usually, scientists prefer to focus on specific features that are meaningful as dictated by their underlying research goal. Potential features detected by the region-growing algorithm, however, are often cluttered and relatively coarse, which might be far from expected. Further, reducing the number of tracked features makes sense from the perspective of computational expense. Therefore, after the potential features set was generated, an iterative process (TODO) of refinement could be applied to refine the result. 

Existing research [27] has detailed the applicability and performance of context-based 3D shape descriptors and their corresponding retrieval methods. Based on the discussion in this work, we can characterize our interactive refinement process as primarily concerned with those computationally efficient yet with significant discriminative power and deformation robust descriptors, such as size, location, or skeletal curvature. 

After specific features have been extracted in a single time step, their evolution can be tracked over time using a prediction-correction approach, which was proved effective and efficient [32]. The prediction of feature's region in space can be made based on the location of features reside in consecutive time steps. Once prediction is made, the actual region could be refined adjusting the surface of the predicted region by first shrink the edge surface points to obtain the mutual region between consecutive time steps, then use of a technique based on region-growing to obtain the actual region.

However, as the size of the data grows such that the original volume data cannot fit into one single processor, a cluster of multiple processors need to be used to process the data in parallel. One challenge of the parallelization of the aforementioned approach lies in that, since volumetric features may span over multiple processors, the global feature descriptions cannot be obtained unless they can be shared and merged in a efficient way, as partial features reside in different nodes will be treated independently. An intuitive way of sharing such feature information is to find the connectivity of features span over multiple nodes, and then integrate them per feature.

\subsection{Creating Partial connectivity graph}
If we partition a large volumetric dataset using a regular grid, it is very likely that some of the feature blobs will be cut off on the boundary surface of its residing block. And for two data blocks that are physically adjacent to each other, their boundary surface should match. Leveraging this property, we can connect a feature resides in adjacent data blocks by comparing their correspondent boundary surface. 

Since the data is distributed, vortex data in adjacent blocks are invisible to the current processor. Though intuitively, exchanging the array of all the voxices on the boundary surface should work for finding the possible matches, we choose to exchange more abstracted data, the min-max coordinate of the boundary box and the geometric centroid of the boundary surface, to reduce the communication load over the network. The min-max coordinates are not optional that they ensure correct connectivity for cases that a sectional surface is surrounded by a concave or hollow surface whose centroid happen to be the same. see _figure_x_. 

A voxel-width "ghost surface" that stores boundary surface belonging to neighboring blocks might also help to achieve voxel-wise matching of partial features. However, for non-post-processing cases where dataset are not pre-determined, maintaining such "ghost surface" requires inter-process communication and thus was considered expensive for data generated in real-time. Instead, a ±1-voxel tolerance was given when matching geometric centroids on the boundary surfaces. That is, if the neighboring centroid is within the eight direct neighbor of a domestic centroids, they are considered a match.

_Some other special cases, though they rarely happen in real world dataset, also need to considered. As shown in _Figure x_, in certain time steps there might exist features with very sharp edge surface right on the boundary, or even more extreme, parttwo parts of features reside in two adjacent data blocks with an sharp gap in shape on the correspondnet boundary surface (_Figure x_), this will also be considered as a single feautre as a whole._

The creation of partial connectivity graph will not introduce extra computational cost as it could be done along with the region growing process. A new edge is appended to the existing graph when a feature touches the block boundary, taking the ID of the target block that shares a same boundary surface as the ending vertex, as well as the global index of the centroids point as edge value.


And the detail algorithm is given in _Table x_.

// TODO
Memory cost, each feature on boundary will use two INTs, 1 as global centroids coordinate, the other as target node number. even if there's 1000 features on the boundary, it won’t cost more than 1000 * 2 * 4 = 8mb to store this graph, neglectable compare to the volume data itself. (But kind of large if used for communication)

\subsection{Creating Global connectivity graph}

\subsubsection{The naive solution}
To obtain the global descriptions of features, partial connectivity information need to be merged together after they were individually created. A naive solution to gather partial connectivity graphs is to send all edges sharing the same ending vertex to that target processor, and to merge edges sent from other processor with local partial connectivity graph. Two edges are merged if 1. their starting and ending vertices are _reversely matched_ and 2. both edge index are located within direct neighbors. Recall that the starting vertex of an edge represents the current processor ID and ending vertex the ID of target processor, whose data block is adjacent to the one that reside in the current processor, and the edge index is encoded by the global coordinate of the geometric centroid on the shared boundary surface. If two edges match to each other, the two partial features share a same boundary surface with the same centroid and bounding region. In other word, these two partial features were cut off when partitioning the original dataset, and should be considered the same feature sharing the same feature ID.

The detail algorithm of this REDUCE process is given in _Table x_.

The naive solution may works well for those dataset whose feature size is small. However if we consider the case that there exist a long curly feature inside the volume and was partitioned evenly over the grid. In order to gather and merge the whole feature, the processors need to talk to its neighbor and spread the edge exchange operation one by one like that in the "telephone" game. This takes O(N) times communication to connect a single feature, where N is the number of total number of processors in the grid. And consequently O(num_feature * N) times communication for all the features within one time step. Also, since one processor has no idea if it will receive any edges from its adjacent processors, it is hard to schedule the communication process.

\subsubsection{The centralized approach}
A possible solution to reduce the number of times required for merging all edges is to employ the master-slave hierarchy by introducing a separate host processor. When the feature extraction process is done, partial connectivity graphs with edges representing how many features have touched the block boundary and where they are located on the surface within each individual processor will be gathered to the host processor. After the GATHER operation is done, i.e. host processor has collected all partial connectivity graphs, the REDUCE operation starts to merge edges from all partial graphs to a single global connectivity graph.

The merit of this centralized solution is that it only requires sending the data only once. Also, a global graph of feature information could be preserved in the host processor, which made it easy for the host to answer connectivity queries directly without pulling information from slave processors a second time. However the drawback of this approach is also obvious. Since all partial graphs preserved in each processor will be sent to one single processor, there exists potential bottleneck, both communicational and computational, on the host processor.

\subsubsection{Two decentralized approaches}
A better solution is to decentralize the merging process from a single host to all processors that are available, as depicted in _Figure x_. After the feature extraction process is done and so does the creation of partial connectivity graph, the ALL_GATHER process starts to exchange all partial connectivity graphs within each processor to all the others. This time, each processor will first collect a full copy of all partial connectivity graphs followed by the same REDUCE process to merge the edges into the concise graph. 

Though the "redundant" host processor is not required when applying for computation resources, this approach does not actually solve the bottleneck problem since now every processor is act like a host while they still need to gather all the partial graphs and try merging them. For real world dataset, however, it is rarely the case that a single feature will span over every processor. In other word, it is unnecessary to gather edges of feature that are not reside in current processor. To reduce the redundant communications with every other processor in the grid, the decentralized approach could be further improved in that we only consider those processors that are directly adjacent to the current one. That is, we only talk to our neighbors and share information with them regardless what is happening outside the community. 

For a partitioned volumetric dataset, there are only 6 direct neighbors for each processor (for processor on the “edge” of the grid there are even less). Instead of gathering partial connectivity graphs from all processors in the grid, only graphs reside in neighboring processors will be exchanged. This could be regarded as a higher level of region growing, starting from one seeding processor and grows to all adjacent processors, exchange and merge partial connectivity graphs in a bread-first fashion until all residing processors of the same feature have been connected. For the worst case that a feature span over all of the processors, it takes O(C∛(|N|)) time to finish the search for connected processors, much faster than the previous O(N) time for all gathering among all processors.

\subsubsection{The hybrid approach}
We can still take a step further to optimize the aforementioned decentralized approach. As volume data evolves over time, the internal features may vary but should not change drastically in size and shape nor location if the time interval for sampling is short enough. Thus we can apply the prediction-correction approach to further reduce the number of times required to complete the connectivity graph. 

Every time (say, ti) when the global connectivity graph is completed, new local communicators will be updated for the next time step (ti+1), with the union of processors that share a same edge with the current processor, as shown in _figure x_. Edges from these processors are required to complete the global connectivity graph anyway, no matter which approach is used. Hence, for these _must-involve_ processors, we apply the _Decentralize-I_ approach, allowing the minimum one-time synchronization to finish gathering all edges necessary for updating the connectivity graph based on the graph created at previous time step (t). Then, the processor-level region growing explained in 3.3.3 is applied to extend the "boundary" of processes, obtaining newly connected processors caused by the evolution of the original volume. Beside, only those edges that are changed and not synchronized will be sent. This makes sure we keep the amount of data being sent over network to the minimum necessity.
The detail algorithm of the hybrid approach is given in _Table x_.

\section{Application}

\subsection{Feature selection and refinement}
Since the feature extraction are done independently within each PE, one of the potential function need to be addressed is that how do they know if their adjacent processor was has the feature to be highlighted and whether it has the other part of the features. Intuitively, sending a message to the adjacent PE whenever a feature was detected touching the surface boundary can do this. But if the target feature spans over multiple PEs, this sending/receiving procedure would take several rounds to end. This is potentially a big problem when the PE/Volume ratio is relatively high such that each feature spans over a lot of PEs. Another problem is that, if a PEs has two selected features whose connectivity info arrives in different rounds, it requestes to calculate twice, which again, will become a problem when PE/Volume ratio is large.

By introducing the global connectivity graph in our approach, whenever part of the feature was selected, the unique feature id will be sent back to host processor and then broadcast to all PEs contains it, and only in one rounds, the selection can be finished.

Based on the coordinate user specified or clicked on the volume rendering result, the residual processor as well as the feature, if the point was included by, could be obtains. Then the host processor can simply broadcast the selected feature id to all those processors who has the partial feature such that they can be highlighted.


\subsection{Feature tracking}
Before you begin to format your paper, first write and save the content as a separate text file. Keep your text and graphic files separate until after the text has been formatted and styled. Do not use hard tabs, and limit use of hard returns to only one return at the end of a paragraph. Do not add any kind of pagination anywhere in the paper. Do not number text heads-the template will do that for you.

Finally, complete content and organizational editing before formatting. Please take note of the following items when proofreading spelling and grammar:

\section{Result}
\subsection{Performance Result}
\subsection{Visualiztion Result}

\section{Conclusion}
In this paper, we proposed a decentralized approach that all feature connectivity information are created and preserved in distributed processors. Traditional approaches perform connectivity test on each processor and subsequently correspond them in a host processor after gathering all or partially merged connectivity information. Our approach does not follow this paradigm. Rather, instead of sending local connectively information back to the host, they are computed and preserved only in processors where correspondent feature resides in. There is no copy of the global feature information preserved in the host processors and it only acts as the interface from where the criterion of feature of interest is broadcasted. By doing this the computation of merging local connectivity information is distributed to the non-host s and it effectively reduces the potential communication bottleneck on the host processor. What's more, there's no need to set a barrier and wait before all connectivity information was send back to the host, thus if one of the features spans over a large number of processors but was not selected by the user, the potentially long computation time for this feature will not be considered. This makes it ideal for an interactive system, where users can select the feature of interest and instantly receive the visual feedback as the feature evolves.

% An example of a floating figure using the graphicx package.
% Note that \label must occur AFTER (or within) \caption.
% For figures, \caption should occur after the \includegraphics.
% Note that IEEEtran v1.7 and later has special internal code that
% is designed to preserve the operation of \label within \caption
% even when the captionsoff option is in effect. However, because
% of issues like this, it may be the safest practice to put all your
% \label just after \caption rather than within \caption{}.
%
% Reminder: the "draftcls" or "draftclsnofoot", not "draft", class
% option should be used if it is desired that the figures are to be
% displayed while in draft mode.
%
%\begin{figure}[!t]
%\centering
%\includegraphics[width=2.5in]{myfigure}
% where an .eps filename suffix will be assumed under latex, 
% and a .pdf suffix will be assumed for pdflatex; or what has been declared
% via \DeclareGraphicsExtensions.
%\caption{Simulation Results}
%\label{fig_sim}
%\end{figure}

% Note that IEEE typically puts floats only at the top, even when this
% results in a large percentage of a column being occupied by floats.


% An example of a double column floating figure using two subfigures.
% (The subfig.sty package must be loaded for this to work.)
% The subfigure \label commands are set within each subfloat command, the
% \label for the overall figure must come after \caption.
% \hfil must be used as a separator to get equal spacing.
% The subfigure.sty package works much the same way, except \subfigure is
% used instead of \subfloat.
%
%\begin{figure*}[!t]
%\centerline{\subfloat[Case I]\includegraphics[width=2.5in]{subfigcase1}%
%\label{fig_first_case}}
%\hfil
%\subfloat[Case II]{\includegraphics[width=2.5in]{subfigcase2}%
%\label{fig_second_case}}}
%\caption{Simulation results}
%\label{fig_sim}
%\end{figure*}
%
% Note that often IEEE papers with subfigures do not employ subfigure
% captions (using the optional argument to \subfloat), but instead will
% reference/describe all of them (a), (b), etc., within the main caption.


% An example of a floating table. Note that, for IEEE style tables, the 
% \caption command should come BEFORE the table. Table text will default to
% \footnotesize as IEEE normally uses this smaller font for tables.
% The \label must come after \caption as always.
%
%\begin{table}[!t]
%% increase table row spacing, adjust to taste
%\renewcommand{\arraystretch}{1.3}
% if using array.sty, it might be a good idea to tweak the value of
% \extrarowheight as needed to properly center the text within the cells
%\caption{An Example of a Table}
%\label{table_example}
%\centering
%% Some packages, such as MDW tools, offer better commands for making tables
%% than the plain LaTeX2e tabular which is used here.
%\begin{tabular}{|c||c|}
%\hline
%One & Two\\
%\hline
%Three & Four\\
%\hline
%\end{tabular}
%\end{table}


% Note that IEEE does not put floats in the very first column - or typically
% anywhere on the first page for that matter. Also, in-text middle ("here")
% positioning is not used. Most IEEE journals/conferences use top floats
% exclusively. Note that, LaTeX2e, unlike IEEE journals/conferences, places
% footnotes above bottom floats. This can be corrected via the \fnbelowfloat
% command of the stfloats package.



\section{Conclusion}
The conclusion goes here. this is more of the conclusion

% conference papers do not normally have an appendix


% use section* for acknowledgement
\section*{Acknowledgment}


The authors would like to thank...
more thanks here


% trigger a \newpage just before the given reference
% number - used to balance the columns on the last page
% adjust value as needed - may need to be readjusted if
% the document is modified later
%\IEEEtriggeratref{8}
% The "triggered" command can be changed if desired:
%\IEEEtriggercmd{\enlargethispage{-5in}}

% references section

% can use a bibliography generated by BibTeX as a .bbl file
% BibTeX documentation can be easily obtained at:
% http://www.ctan.org/tex-archive/biblio/bibtex/contrib/doc/
% The IEEEtran BibTeX style support page is at:
% http://www.michaelshell.org/tex/ieeetran/bibtex/
%\bibliographystyle{IEEEtran}
% argument is your BibTeX string definitions and bibliography database(s)
%\bibliography{IEEEabrv,../bib/paper}
%
% <OR> manually copy in the resultant .bbl file
% set second argument of \begin to the number of references
% (used to reserve space for the reference number labels box)
\begin{thebibliography}{1}

\bibitem{IEEEhowto:kopka}
H.~Kopka and P.~W. Daly, \emph{A Guide to \LaTeX}, 3rd~ed.\hskip 1em plus
  0.5em minus 0.4em\relax Harlow, England: Addison-Wesley, 1999.

\end{thebibliography}




% that's all folks
\end{document}


