Large-scale time-varying volumetric data set can take terabytes to petabytes of storage space to store and process. One promising approach is to process the data in parallel in situ or batch, extract features of interest and analyze only those features, to reduce required memory space by several orders of magnitude for the following visualization tasks. However, extracting volumetric features in parallel is a non-trivial task as features might span over multiple processors, and local partial features can be only visible within their own processor. In this paper, we discuss how to generate and maintain connectivity information of features across different processors. Based on this connectivity information, partial features can be integrated, which makes possible for the large-scale feature extraction and tracking in parallel. We demonstrate the effectiveness and scalability of different approaches on two data sets, and discuss the pros and cons of each approach.