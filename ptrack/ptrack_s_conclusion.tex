\section{Conclusion}
\textcolor{red}{To be revised}
In this paper, we presented a decentralized approach that all feature connectivity information are created and preserved among distributed processors. Traditional approaches perform connectivity test on each processor and subsequently correspond them in a host processor after gathering all or partially merged connectivity information. Our approach does not follow this paradigm. Rather, instead being sent back to the host, the local connectivity information are computed and preserved only in the local processor.
There is no copy of the global feature information preserved in the host, and the host only acts as the interface from where the criterion of feature of interest is broadcast. In this way, the computation of merging local connectivity information is distributed to the slaves, which can effectively remove the potential communication bottleneck on the host processor.
Moreover, there's no need to set a barrier and wait for all connectivity information being sent back to the host, thus if one of the features spans over a large number of processors but was not selected by the user, the potentially long computation time for this feature will not be considered. This makes it ideal for an interactive system, where users can select the feature of interest and instantly receive the visual feedback as the feature evolves.